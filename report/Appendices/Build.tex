\chapter{How to build the project}
\label{AppendixB}
\lhead{Appendix B. \emph{Build project}} 

\section{Database}

Before deploying the NIPEN system it is necessary to set up the database.
We are running a MySQL server on a ubuntu machine, the version we are using is \textit{5.5.32-0ubuntu0.13.04.1}.
After the MySQL server is installed the file \textit{database.sql} must be used to set up the database.
It contains all the database tables we are using in our project.
The project folder in \textit{nipen\textbackslash src\textbackslash main\textbackslash resources\textbackslash database}, contains a file called \textit{Spring-Datasource.xml}.
This file contains the database location, user and password.
This must be set up according to the configurations of the MySQL database.

\section{NIPEN and the front-end}

Our back-end and front-end is running on an Apache Tomcat Server 7.0.35, it can be downloaded from the following link: \href{http://tomcat.apache.org/download-70.cgi}{http://tomcat.apache.org/download-70.cgi}.
By default the manager page of tomcat can be found at the following URL $<$server address$>$/manager.
On that page it is possible to upload a WAR file on to the server.

To create a WAR file of NIPEN and the front-end, Java compiler and maven must be installed on the system.
The Java compiler version we are using is 1.7.0\_25.
We are using Maven 3.1.0, and it can be downloaded on the following page: \href{http://maven.apache.org/download.cgi}{http://maven.apache.org/download.cgi}.
When maven is installed and configured on the system, we can create a war file of the NIPEN project.
First we need to locate the \textit{pom.xml} file in the NIPEN folder.
After that we need to open a command line in that folder and run the command: \begin{verbatim}
mvn package
\end{verbatim}
This will create a target folder that contains the file \textit{nipen.war}.
When deploying this file on the tomcat server both the back-end (NIPEN) and the front-end (the web page) will be deployed on the server.

\section{HealthVault Integration Service}

This is deployed in the same way as NIPEN, except that this does not need a database.
Find the folder of the web service that contains the pom file, then run \textit{mvn package}.
The war file can then be uploaded on to the tomcat server.

\section{Android applications}

To install the android applications it is needed to copy the two apk files on to an android device.
One apk file for the heart rate application and one for the weight application.
The applications can then be installed on the android device through the default package installer on the phone.
However, since these applications are not downloaded from Google Play, a setting needs to be set on the phone to allow instalment of files from unknown sources.
This setting is usually found under application settings on the android device, and is called \textit{Unknown Sources}.