\chapter{Test results}
\label{AppendixG}
\lhead{Appendix G. \emph{Test results}}

\section{Functional tests}

\begin{table}
\begin{center}
\begin{tabular}{ | l | p{10cm} | }
	\hline
	\textbf{Test}	&	\textbf{ID 1} \\
	\hline\noalign{\smallskip}\noalign{\smallskip}\hline
	Application name	& NIPEN \\
	Name				& Heart rate REST (read) \\
	Requirement			& \hyperref[table:reqip]{FIP1} \\
	Description			& Test the REST endpoint for receiving heart rate data models \\
	Preconditions		& 	\par The IP is deployed and running on a server machine 
							\par The test is run on the server machine or alternatively
							one that has access to the server and whose address is replaced to the
							string \verb|localhost| in the test.
							\par The machine on which the test is run has the \verb|curl| program installed.
							\par At least one heart rate measurement has been stored on the IP \\
	Steps 				&	1. Run the following:
							\begin{verbatim}
							curl localhost:8080/nipen/human/api/heart_rates
							\end{verbatim}
							\\
	Postconditions		& A JSON valid, comma separated list of heart rate data models consistent with 
							entries in the databased hosted on the server. \\
	Results				& An array of heart rate JSON strings is returned. 
						  The values are consistent with the values on the database. 
						  Hence, the test was successful. \\
	Comments			& - \\
	Status				& OK \\
	Tester				& Sebastian Zalewski \\
	Date				& 08-11-2013 \\
	\hline
\end{tabular}
\end{center}
\end{table}

\begin{table}
\begin{center}
\begin{tabular}{ | l | p{10cm} | }
	\hline
	\textbf{Test}	&	\textbf{ID 2} \\
	\hline\noalign{\smallskip}\noalign{\smallskip}\hline
	Application name	& NIPEN \\
	Name				& Weight REST (read) \\
	Requirement			& \hyperref[table:reqip]{FIP2} \\
	Description			& Test the REST endpoint for receiving weight data models \\
	Preconditions		&	\par The IP is deployed and running on a server machine
							\par The test is run on the server machine or alternatively
							one that has access to the server and whose address is replaced to the
							string \verb|localhost| in the test.
							\par The machine on which the test is run has the \verb|curl| program installed.
							\par At least one weight measurement has been stored on the IP \\
	Steps 				&	1. Run the following:
							\begin{verbatim}
							curl localhost:8080/nipen/human/api/weights
							\end{verbatim}
							\\
	Postconditions		& A JSON valid, comma separated list of heart rate data models consistent with 
							entries in the database hosted on the server. \\
	Results				& An array of weight JSON strings is returned. 
						  The values are consistent with the values on the database. 
						  Hence, the test was successful. \\
	Comments			& - \\
	Status				& OK \\
	Tester				& Sebastian Zalewski \\
	Date				& 08-11-2013 \\
	\hline
\end{tabular}
\end{center}
\end{table}

\begin{table}
\begin{center}
\begin{tabular}{ | l | p{10cm} | }
	\hline
	\textbf{Test}	&	\textbf{ID 3} \\
	\hline\noalign{\smallskip}\noalign{\smallskip}\hline
	Application name	& NIPEN \\
	Name				& Heart rate REST (write) \\
	Requirement			& \hyperref[table:reqip]{FIP3} \\
	Description			& Test the REST endpoint for requesting heart rate data models \\
	Preconditions		&	\par The IP is deployed and running on a server machine
							\par The test is run on the server machine or alternatively
							one that has access to the server and whose address is replaced to the
							string \verb|localhost| in the test.
							\par The machine on which the test is run has the \verb|curl| program installed. \\
	Steps 				&	1. Run the following \begin{verbatim}
							curl -X POST -H "Content-Type: application/json" 
							-d "{\"userId\":1,\"timestamp\":
							\"2013-11-08 17:33:05\",\"value\":60,
							\"unit\":\"bpm\"}" 
							localhost:8080/nipen/api/human/heart_rate
							\end{verbatim} \\
	Postconditions		& A database entry coherent with the JSON data submitted is created on the server. \\
	Results				& A new entry on the database is created with the sent values.
						  Hence, the test was successful. \\
	Comments			& - \\
	Status				& OK \\
	Tester				& Sebastian Zalewski \\
	Date				& 08-11-2013 \\
	\hline
\end{tabular}
\end{center}
\end{table}

\begin{table}
\begin{center}
\begin{tabular}{ | l | p{10cm} | }
	\hline
	\textbf{Test}	&	\textbf{ID 4} \\
	\hline\noalign{\smallskip}\noalign{\smallskip}\hline
	Application name	& NIPEN \\
	Name				& Weight REST (write) \\
	Requirement			& \hyperref[table:reqip]{FIP4} \\
	Description			& Test the REST endpoint for requeting weight data models \\
	Preconditions		&	\par The IP is deployed and running on a server machine
							\par The test is run on the server machine or alternatively
							one that has access to the server and whose address is replaced to the
							string \verb|localhost| in the test.
							\par The machine on which the test is run has the \verb|curl| program installed. \\
	Steps 				&	1. Run the following \begin{verbatim}
							curl -X POST -H "Content-Type: application/json" 
							-d "{\"userId\":1,\"timestamp\":
							\"2013-11-08 17:45:05\",\"value\":89,
							\"unit\":\"kg\"}" 
							localhost:8080/nipen/api/human/weight
							\end{verbatim} \\
	Postconditions		& A database entry coherent with the JSON data submitted is created on the server. \\
	Results				& A new entry on the database is created with the sent values.
						  Hence, the test was successful. \\
	Comments			& - \\
	Status				& OK \\
	Tester				& Sebastian Zalewski \\
	Date				& 08-11-2013 \\
	\hline
\end{tabular}
\end{center}
\end{table}

\begin{table}
\begin{center}
\begin{tabular}{ | l | p{10cm} | }
	\hline
	\textbf{Test}	&	\textbf{ID 5} \\
	\hline\noalign{\smallskip}\noalign{\smallskip}\hline
	Application name	& Heart Rate Application \\
	Name				& Heart rate measurement \\
	Requirement			& \hyperref[table:reqheartrate]{FHR1} \\
	Description			& Test the heart rate measurement functionality \\
	Preconditions		& The application has started \\
	Steps 				&	\par 1. User holds his finger on the camera applying a slight pressure \\
	Postconditions		&	\par 1. The icon on the left side of the screen is blinking 
							\par 2. A measurement appears after 4 seconds at most \\
	Results				& The result was a heart measurement of 77, which is accurate.
						  However, the measurement didn't appear until after 6 or 7 seconds.
						  This might be because of wrong finger placement, and may also vary from phone to phone.
						  The measurement was quite accurate and hence the test was successful. \\
	Comments			&	The measurement is expected to be rough.
							Any value between 60 and 100 is okay as long as it actually varies slightly based
							on the tester's perceived heart rate.  \\
	Status				& OK \\
	Tester				& Sebastian Zalewski \\
	Date				& 08-11-2013 \\
	\hline
\end{tabular}
\end{center}
\end{table}

\begin{table}
\begin{center}
\begin{tabular}{ | l | p{10cm} | }
	\hline
	\textbf{Test}	&	\textbf{ID 6} \\
	\hline\noalign{\smallskip}\noalign{\smallskip}\hline
	Application name	& Heart Rate Application \\
	Name				& Send heart rate \\
	Requirement			& \hyperref[table:reqheartrate]{FHR3} \\
	Description			& Test the heart rate application interoperability \\
	Preconditions		&	\par 1. The application has started
							\par 2. A heart rate measurement has been acquired \\
	Steps 				&	\par 1. User presses \textbf{Send} button \\
	Postconditions		&	\par 1. The application shows a toast
							\par 2. The measurement is acquired by the Integration Platform \\
	Results				& The measurement was sent to the front-end and contained the right values.
						  The test was successful. \\
	Comments			& - \\
	Status				& OK \\
	Tester				& Sebastian Zalewski \\
	Date				& 08-11-2013 \\
	\hline
\end{tabular}
\end{center}
\end{table}


\begin{table}
\begin{center}
\begin{tabular}{ | l | p{10cm} | }
	\hline
	\textbf{Test}	&	\textbf{ID 7} \\
	\hline\noalign{\smallskip}\noalign{\smallskip}\hline
	Application name	& Weigh Application \\
	Name				& Data fetching from HealthVault\\
	Requirement			& \hyperref[table:reqweight]{FHV1} \\
	Description			& Test HealthVault connectivity \\
	Preconditions		&	\par 1. The application has started
							\par 2. The user has authenticated to HealthVault \\
	Steps 				&	\par 1. User clicks on the \textit{Fetch} button \\
	Postconditions		&	\par 1. The data is correctly fetched from HealthVault
							\par 2. A list is populated with the data \\
	Results				& The values appearing on the app are corresponding to the values on HealthVault.
						  Thus the test was successful. \\
	Comments			& - \\
	Status				& OK \\
	Tester				& Sebastian Zalewski \\
	Date				& 08-11-2013 \\
	\hline
\end{tabular}
\end{center}
\end{table}

\begin{table}
\begin{center}
\begin{tabular}{ | l | p{10cm} | }
	\hline
	\textbf{Test}	&	\textbf{ID 8} \\
	\hline\noalign{\smallskip}\noalign{\smallskip}\hline
	Application name	& Weight Application \\
	Name				& Sending data to HealthVault through app \\
	Requirement			& \hyperref[table:reqweight]{FHV3} \\
	Description			& Test the HealthVault application data forwarding functionality \\
	Preconditions		&	\par 1. The application has started
							\par 2. The application has fetched some data from HealthVault \\
	Steps 				&	\par 1. User enters a weight value
							\par 2. User clicks on the \textit{Send} button \\
	Postconditions		&	\par 1. A success message is showed
							\par 2. Data is correctly acquired by the integration platform \\
	Results				& The data sent is acquired by the integration platform, with the right values.
						  This is seen on the front-end and in the database.
						  The test was successful. \\
	Comments			& - \\
	Status				& OK \\
	Tester				& Sebastian Zalewski \\
	Date				& 08-11-2013 \\
	\hline
\end{tabular}
\end{center}
\end{table}

\begin{table}
\begin{center}
\begin{tabular}{ | l | p{10cm} | }
	\hline
	\textbf{Test}	&	\textbf{ID 9} \\
	\hline\noalign{\smallskip}\noalign{\smallskip}\hline
	Application name	& HealthVault Integration Service \\
	Name				& Sending data to HealthVault through web service \\
	Requirement			& \hyperref[table:reqwebservice]{FHIS1} \\
	Description			& Test if the web service i capable of sending values to HealthVault \\
	Preconditions		&	\par 1. The user has logged in to HealthVault through the web service \\
	Steps 				&	\par 1. Enter a value into the weight field
							\par 2. Click on submit to HealthVault button \\
	Postconditions		&	\par 1. The submitted value is stored on HealthVault \\
	Results				& The value entered is registered on HealthVault.
						  The test was successful. \\
	Comments			& - \\
	Status				& OK \\
	Tester				& Sebastian Zalewski \\
	Date				& 08-11-2013 \\
	\hline
\end{tabular}
\end{center}
\end{table}

\begin{table}
\begin{center}
\begin{tabular}{ | l | p{10cm} | }
	\hline
	\textbf{Test}	&	\textbf{ID 9} \\
	\hline\noalign{\smallskip}\noalign{\smallskip}\hline
	Application name	& HealthVault Integration Service \\
	Name				& Polling service \\
	Requirement			& \hyperref[table:reqwebservice]{FHIS2} and \hyperref[table:reqwebservice]{FHIS2} \\
	Description			& Test if new weight measurements stored on HealthVault are sent to NIPEN \\
	Preconditions		&	\par 1. The user has logged in to HealthVault through the web service
							\par 2. The polling service is disabled \\
	Steps 				&	\par 1. User clicks on \textit{Enable} button, to start the polling 
							\par 2. User adds a value into HealthVault \\
	Postconditions		&	\par 1. The value is sent to the integration platform
							\par 2. The sent value can be viewed on the front-end \\
	Results				& The value that was entered on HealthVault is visible on the front-end.
						  Hence, the test was successful. \\
	Comments			& - \\
	Status				& OK \\
	Tester				& Sebastian Zalewski \\
	Date				& 08-11-2013 \\
	\hline
\end{tabular}
\end{center}
\end{table}