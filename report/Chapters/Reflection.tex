
\chapter{Reflection}
\lhead{Chapter 16. \emph{Reflection}}

\section{Suggestions for course improvements}
In all courses there are opportunities for improvement.
In this section we will discuss what we thought could be better, clearer and more efficient.

Firstly, we as a group felt that there was a lot of information we didn't get from the start. 
The first day of the course was the day we meet the client and our group members had not prepared for this.
The first couple of weeks there was a lack of information about what was expected and what was to be delivered.
We had to figure out a lot along the way.

This project would definitively improve with having more group members. 
We ended up being only three people while this project was really meant for atleast five people.
That left us with a lot of tasks and a lot of extra work to do.
Especially writing the report and developing the project would be a lot better with more members and work hours. 
We are pleased with the result of our project but it had potential to be a lot better with more hours of work which would only come from more group members. 

There were also some lectures throughout the semester. 
The times of these lectures were put at random times and not in the specified course time slots.
This lead to a lot of conflicts between course.
We advice that the course leaders try their best to plan lectures in the appointed timeslots so this won't happen in future years. 

We were early in reserving rooms for this course so we were maybe lucky in getting a room that we could work in this semester. 
Maybe other groups were not as lucky. 
The reserving of the rooms took some time for us to fix. 
It would be better for us to get a room appointed to our group that we could work in every week. 




\iffalse
Our NIPEN implementation and the belonging applications are mostly a proof of concept. 
The most important task of the client to do next will be to figure out if there is an interest in Helsenorge for this type of system.
The most important factor for this system is the value it delivers to the educated medical professionals. 
If this is something that gives the medical professionals additional value and makes it easier for them to understand the patiens health. 
A cost and benefit analysis of this system needs to be done to analyse the total value this system will give will be an important factor in deciding where to go next.
The thought is that citizens can collect all types of health data and that even though their measurements might be imprecise that quantity of health data will overall improve the quality.
This system is definitely possible to implement at a larger scale at a high cost.
The toughest part will be to convince the medical professionals to start using new methods and systems. 
Many will be pessimistic for this kind of system knowing that the data at some degree could not be dependent upon.
The important part to consider then is that even though specific data might be inaccurate it will be a lot easier to analyse trends and get a sense of the citizens habits. 

The value of this system can be hard to measure but already many people today do these kinds of measurements and the adaption is increasing. 
Earlier this year Pew Research Center’s Internet \& American Life Project released their findings of the role of Internet and technology in health and wellness. 
Their report, Tracking for Health can be found here http://pewinternet.org/Reports/2013/Tracking-for-Health.aspx, is focused on how people self-track.
In the research paper they found that 7 out of 10 adults track their health.
While 1 in 5 use technology to log this. 
What is important to note from the findings they did is that over half of those who keep a record of their health indicate that their tracking and recordkeeping has changed their approach to health.
The conclusion that can be taken from this is that the act of tracking alone affects the overall health and mindset of the citizen. 

\subsection{Third-party applications}

We developed three applications all interacting with our implementation of NIPEN. 
The idea behind making a portal like this is to open the API up to all developers so the NIP can be interoperable between all platforms, systems, applications and users. 
The ideal goal is to have a platform that reaches all types of users. 
The total cost of the project can be lowered by this because support for the system can be done by the developers of the different third-party applications. 
It is also possible for some developers to develop proxies for popular third-party applications.


%%%%
We are glad we managed to implement every requirement 
meaning we made a realistic assumptions of what we could accomplish in the given time with the members we had.
%Although some of the syncronisation did not work ideeal it was not cause by our system but by third-party systems not working ideally. 
This project is an interesting idea and finding a way to unify the collection of health data of citizens will lead to a better understanding of citizen health and to an \iffalse easier\fi overview of what can be improved for a better quality of life. 
\fi