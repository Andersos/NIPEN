\chapter{Preliminary Studies}
\label{Preliminary Studies}
\lhead{Chapter 5. \emph{Preliminary Studies}}

This chapter contains

3.1 dev metho
3.1.1 waterfall
3.1.2 scrum
3.2 existing solutions
3.2.1 HealthVault
3.2.2 open ehealth
3.2.3 human api
3.3 tech
3.3.1 server
3.3.2 database
3.3.3 web page
3.3.4 android
3.4 testing


%-----------
% SECTION 1
%-----------

\section{Development Methodology}

TODO

%-----------------------------------
%	SUBSECTION 1
%-----------------------------------
\subsection{Waterfall Model}

The waterfall model is a software development process where each task is performed in a sequential order.
Before moving to the next phase the preceding task needs to be finished.
The progress of the project is seen as flowing downwards through the different phases, hence the name waterfall.
In the original model the phases consisted of seven different tasks:

\begin{enumerate}
\item Requirements specification
\item Design
\item Construction (implementation or coding)
\item Integration
\item Testing and debugging
\item Installation
\item Maintenance
\end{enumerate}

Because each phase needs to be perfected and completed before moving to the next phase, this brings up some difficulties if the requirements were to change during the development process. 
However the model is easily understandable, structured, and disciplined. 
All the phases are divided into different sections, and this makes it easier to understand the progress of the project.
In practice it can be very hard to adapt to this kind of development model. 
It can be hard for a system designer to predict future implementation difficulties of a type of design, hence the design of the system may change during the process.
Another problem is that the customer is not always sure about the system requirements, and often will the customer change them during the development.

%-----------------------------------
%	SUBSECTION 2
%-----------------------------------

\subsection{SCRUM Model}

TODO

%-----------------------------------
%	SUBSECTION 3
%-----------------------------------
\subsection{Conclusion}

TODO

%----------------------------------------------------------------------------------------
%	SECTION 2
%----------------------------------------------------------------------------------------

\section{Existing Solutions}

TODO

%-----------------------------------
%	SUBSECTION 1
%-----------------------------------
\subsection{HealthVault}

TODO

%-----------------------------------
%	SUBSECTION 2
%-----------------------------------
\subsection{Open eHealth Foundation}

TODO

%-----------------------------------
%	SUBSECTION 3
%-----------------------------------
\subsection{human/api}

The human API is a platform for human health data. 
They have an API that contains multiple different well defined JSON strings for different kinds of human related data.
Each JSON string contains all the necessary information that is needed to represent each type of health data.
For example heart rate is defined by an id, user id, time, value and unit in the following way:

\begin{verbatim}
{
  "id": "string",
  "userId": "string",
  "time": "date",
  "value": "int",
  "unit": "string"
}
\end{verbatim}

%-----------------------------------
%	SUBSECTION 4
%-----------------------------------
\subsection{Conclusion}

TODO

%----------------------------------------------------------------------------------------
%	SECTION 3
%----------------------------------------------------------------------------------------

\section{Technologies}

TODO

%-----------------------------------
%	SUBSECTION 1
%-----------------------------------
\subsection{Server}

Java \\
Spring Framework \\
TODO

Apache Tomcat Server \\ 
TODO

%-----------------------------------
%	SUBSECTION 3
%-----------------------------------
\subsection{Database}

\textbf{MySQL}

TODO

%-----------------------------------
%	SUBSECTION 4
%-----------------------------------
\subsection{Web Page}

\textbf{HTML5}

HTML is the standard World Wide Web's markup language.
It is used to structure and visualize web pages on the internet.

\textbf{CSS3}

CSS describes the look and format of a document written in HTML.

\textbf{Javascript}

TODO

\textbf{JQuery}

TODO

\textbf{Chart.js}

Chart.js is a Javascript library for creating graphs TODO

%-----------------------------------
%	SUBSECTION 5
%-----------------------------------
\subsection{Mobile Technologies}

Android SDK \\
TODO

%-----------------------------------
%	SUBSECTION 6
%-----------------------------------
\subsection{Conclusion}

TODO

%----------------------------------------------------------------------------------------
%	SECTION 4
%----------------------------------------------------------------------------------------

\section{Testing}

TODO

%-----------------------------------
%	SUBSECTION 2
%-----------------------------------
\subsection{Conclusion}

%----------------------------------------------------------------------------------------
%	SECTION 5
%----------------------------------------------------------------------------------------

\section{Summary}

TODO
