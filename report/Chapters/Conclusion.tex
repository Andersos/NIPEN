\chapter{Conclusion}
\label{ch:conclusion}
\lhead{Chapter 14. \emph{Conclusion}} 

In this chapter we provide a conclusion for this work and outline some
starting points for further work.

\section{Conclusion}

We were definitely happy with the project's task which led us to study trending topics
like eHealth and self-quantification. Although we had no previous experiences in designing and developing a web API,
we found it interesting and enjoyable. This resulted in a proactive participation by the group
in the improvement and proposal of requirements, a process which in turn made the project even more interesting
for us to work on.

Preliminary studies have shown a number of existing solutions



Our architectural design choices for the integration platform make it well suited to be scaled up.
The integration platform in fact, consists of a web application and a database which can be deployed separately,
reducing the amount of resources needed on a single machine.
The same applies to the front-end as well because it interacts with the integration platform using its API just like
any other application and therefore does not rely on being deployed on the same machine.
Furthermore, because we are not employing a classical SOAP client-server architecture but a more lean, REST-like one,
the implementation can be considered lightweight in term of resources and it is easier for application developers to interact with.

\section{Further work}

There are certainly many limitation to our product, as described in \ref{subsec:limitations}, which we believe should be
the focus of further work. In this section we present our suggestions on how to overcome said limitations.

\textbf{API design}
A functional system should have a much more extensive and complete API which should be
thoroughly documented and accompanied by a number of examples and libraries which
would contribute to reduce the development effort for third-parties.
Data models should be complete and able to represent a meaningful amount of information,
and possibly exhibit some degree of interoperability with other established standards
such as HL7\footnote{www.hl7.org}, DICOM and CEN's TC/251.

\textbf{Deployment}
The actual architecture makes it possible with some slight refactoring to simply separate
the codebase of the backend and the frontend and deploy them individually on separate machines.
Also, connecting to a database on a different machine can be achieved by simply
modifying the connection address.

\iffalse \textbf{Authentication} \fi


\textbf{Authorization}
Our suggested approach is to implement an OAuth\footnote{www.oauth.net} based authorization mechanism
for third-parties applications. Users should be able to allow different third party applications limited access
to their account, for example only to weights measurements.
%It should not be a problem to apply OAuth to NIPEN.
%The functionality that NIPEN needs to implement is a capability of registering third party applications.
Whenever third party applications are registered, they should be associated on the system with a client ID which
should then be returned to the the application itself.
When a user allows access to an application an access token representing the level of authorization should be sent
to the specified application, so it gets partial access to the users NIPEN account.
At any time users should be able to remove the applications access to NIPEN.

A summary of how this would work is given in the example below, which illustrates how \iffalse the granting of
permission\fi authorization would work with our heart rate application:

\begin{enumerate}
\item User starts the heart rate application.
\item User wants to grant the application access to push heart rate measurements to NIPEN.
\item User is redirected to the website of NIPEN.
\item User logs in to NIPEN (with e.g. BankID).
\item NIPEN asks the user if he wants to grant the heart rate application access to push heart rate data into NIPEN.
\item If user agrees, an access token is sent to the heart rate application.
\end{enumerate}

With OAuth we should also expand the JSON messages that are sent to NIPEN.
They should now contain a client ID and an access token. The client ID is used to identify the application
\iffalse that is pushing values to NIPEN\fi, while the access token
is used for authorization. A JSON string for a heart rate measurement would now look like this:

\begin{lstlisting}[language=json]
{
	"clientID": 489431,
	"accessToken": "safDFSAadsffsasdFDSewfaDSFAdsfaewRETrehhgreeErw",
	"userId":453,
	"timestamp":"2013-11-05 12:12:38.0",
	"value":66,
	"unit":"bpm"
}
\end{lstlisting}

This information should be sent in an encrypted format by using https.

NIPEN will now be able to identify what application is sending data, and is also capable of checking if the application is authorized.
The user is also able to grant other applications partial access to his/hers account, without giving these applications their credentials.
Hence, this should be a good way of handling third party applications in NIPEN.
