\chapter{Results and conclusion}
\label{ch:conclusion}
\lhead{Chapter 16. \emph{Conclusion}} 

%This chapter discusses our results.

%\section{Results}

This project involved designing and developing a number of interoperable systems
consisting of:
\begin{itemize}
\item an integration platform
\item a web front-end %using a combination of HTML, CSS, JS, jQuery and AJAX
\item two Android applications
\item a web application using JSP
\end{itemize}
to analyse the feasibility for a national scale eHealth project to leverage
the wide availability of devices like smartphones and other proprietary solutions like
HealthVault.

We have designed and implemented an integration platform as a Spring based web
application exposing a JSON based API through which is possibile to exchange data models
representing health measurements such as heart rate and weight.
The web application is deployed using Apache Tomcat and implements
persistency using a MySQL database.

\clearpage
The remaining systems have been developed to demonstrate:
\begin{enumerate}[a.]
\item the functionality of the integration platform by using its API to store and retrieve health measurements
\item the feasibility of integrating with already existing third-party solutions such as Microsoft's HealthVault
\item the difficulty for third-party developers to use the API to make they applications interoperable with the integration platform
\end{enumerate}

Our results outline not only the possibility to leverage handheld devices, which we have demonstrated using
the \textit{Heart rate} application, but also the easiness with which third-party developers can make
their application interoperable with our implementation of the integration platform.
In fact, this has been accomplished with less than 100 LOC.
Additionally, our \textit{Weight} application and \textit{HealthVault integration service}
have shown how it is possible to integrate information from other solution provides as well.

Our architectural design choices for the integration platform make it well suited to be scaled up.
The integration platform in fact, consists of a web application and a database which can
be deployed separately, reducing the amount of resources needed on a single machine.
The same applies to the front-end as well because it interacts with the integration
platform using its API just like any other application and therefore does not rely on being deployed
on the same machine.
Furthermore, because we are not employing a classical SOAP client-server architecture but
a more lean, REST-like one, the implementation can be considered lightweight in term of resources
and it is easier for application developers to interact with.


%%% this is not further work, it looks like a conclusion.
%%% it should be reworked and moved to the 'summary' subsection below or even better in the
%%% evaluation chapter together with the 'third-party applications' subsection
\iffalse
Our NIPEN implementation and the belonging applications are mostly a proof of concept. 
The most important task of the client to do next will be to figure out if there is an interest in Helsenorge for this type of system.
The most important factor for this system is the value it delivers to the educated medical professionals. 
If this is something that gives the medical professionals additional value and makes it easier for them to understand the patiens health. 
A cost and benefit analysis of this system needs to be done to analyse the total value this system will give will be an important factor in deciding where to go next.
The thought is that citizens can collect all types of health data and that even though their measurements might be imprecise that quantity of health data will overall improve the quality.
This system is definitely possible to implement at a larger scale at a high cost.
The toughest part will be to convince the medical professionals to start using new methods and systems. 
Many will be pessimistic for this kind of system knowing that the data at some degree could not be dependent upon.
The important part to consider then is that even though specific data might be inaccurate it will be a lot easier to analyse trends and get a sense of the citizens habits. 

The value of this system can be hard to measure but already many people today do these kinds of measurements and the adaption is increasing. 
Earlier this year Pew Research Center’s Internet \& American Life Project released their findings of the role of Internet and technology in health and wellness. 
Their report, Tracking for Health can be found here http://pewinternet.org/Reports/2013/Tracking-for-Health.aspx, is focused on how people self-track.
In the research paper they found that 7 out of 10 adults track their health.
While 1 in 5 use technology to log this. 
What is important to note from the findings they did is that over half of those who keep a record of their health indicate that their tracking and recordkeeping has changed their approach to health.
The conclusion that can be taken from this is that the act of tracking alone affects the overall health and mindset of the citizen. 

\subsection{Third-party applications}

We developed three applications all interacting with our implementation of NIPEN. 
The idea behind making a portal like this is to open the API up to all developers so the NIP can be interoperable between all platforms, systems, applications and users. 
The ideal goal is to have a platform that reaches all types of users. 
The total cost of the project can be lowered by this because support for the system can be done by the developers of the different third-party applications. 
It is also possible for some developers to develop proxies for popular third-party applications.
\fi

%\section{Summary}



%%% this stuff should go in the evaluation chapter
\iffalse
We are glad we managed to implement every requirement 
meaning we made a realistic assumptions of what we could accomplish in the given time with the members we had.
%Although some of the syncronisation did not work ideeal it was not cause by our system but by third-party systems not working ideally. 
This project is an interesting idea and finding a way to unify the collection of health data of citizens will lead to a better understanding of citizen health and to an \iffalse easier\fi overview of what can be improved for a better quality of life. 
\fi