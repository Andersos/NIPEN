\chapter{Conclusion}
\label{ch:conclusion}
\lhead{Chapter 14. \emph{Conclusion}} 

In this chapter we provide a conclusion for this work and outline some
starting points for further work.

\section{Conclusion}

In this document we described the design and implementation of a national integration
platform prototype and a number of interoperable systems.

Preliminary studies have shown a number of similar solutions which have certainly influenced
this work and our design choices. \verb|human/api| provided a good starting point for API data models,
but provided restrictions on the sources it could fetch information from.
HealthVault provided a good example of how we envisioned our product, with the exception
of a few details e.g. the language used for data models.

Our work has outlined some of the most essential functionality an integration platform
of this kind could offer and assessed its possibility to leverage proprietary
third-party solutions and modern hand-held devices such as smart-phones.

We are confident that our client will find our work valuable for the insight it can provide
him on the topic and for the use he can make of it in order to stimulate further research
and development in a field which we believe will improve the quality of healthcare in years to come.

\section{Further work}

There are certainly limitations to our product, as described in \ref{subsec:limitations}, which we believe should be
the focus of further work. In this section we present our suggestions on how to overcome said limitations.

\textbf{API design}\newline
A functional system should have a much more extensive and complete API which should be
thoroughly documented and accompanied by a number of examples and libraries which
would contribute to reduce the development effort for third-parties.
Data models should be complete, able to represent a meaningful amount of information
and possibly exhibit some degree of interoperability with other established standards
such as HL7\footnote{See www.hl7.org}, DICOM, CEN's TC/251.


\textbf{API security concerns}\newline
Users should be able to allow certain third party applications to access
their data, or parts of it e.g. only to weights measurements.
In order to do so the system shall have a mean to authenticate third-party applications
and check their authorization to access a set of user's data.

Our suggested approach to implement an authorization mechanism for third-parties
applications is to use OAuth\footnote{See www.oauth.net}.
To authenticate third-party applications, they should be registered on the integration
platform by their developers and granted a secret client-ID which is to be sent along every
request being made. When an authenticated application request access to user's data, the user
is prompted to authorize the application. Upon successful authorization by the user, the application
is provided a token representing the level of authorization granted, e.g. what kind of information
it is allowed access to. Of course, all communication should be encrypted using HTTPS.

A summary of how this would work is given in the example below, which illustrates how this
process would work with our heart rate application:

\begin{enumerate}[1.]
\item The developers register the application an receive a client-ID
\item User starts the heart rate application
\item The application redirects the user to NIPEN's webpage
\item NIPEN authenticates the application against its client-ID
\item The user authenticates himself on NIPEN's website (e.g. using BankID)
\item The user authorizes the application, granting it access to some specific health information
\item An access token is sent to the heart rate application
\end{enumerate}

This approach involves sending along two additional parameters, client-ID and the access token,
with every request. This implies that JSON data models should be review to contain both,
see below an example.

\begin{lstlisting}[language=json]
{
	"clientID": 489431,
	"accessToken": "safDFSAadsffsasdFDSewfaDSFAdsfaewRETrehhgreeErw",
	"userId":453,
	"timestamp":"2013-11-05 12:12:38.0",
	"value":66,
	"unit":"bpm"
}
\end{lstlisting}


OAuth is a solid solution for securing APIs and is used by major companies such as:
Facebook, Twitter, Microsoft, Google, Amazon\ldots
We invite readers to lookup these companies API developers pages for more
informations.

\textbf{Deployment}\newline
The actual architecture makes it possible with some slight refactoring to simply separate
the code-base of the back-end and the front-end and deploy them individually on separate machines.
Also, connecting to a database on a different machine can be achieved by simply
modifying the connection address.