\chapter{Evaluation} 
\label{ch:evaluation}

\lhead{Chapter 13. \emph{Evaluation}}

In this chapter we evaluate various aspects of this project, discussing what we
have learned, what went well and what not.
We outline our results, the status of product's requirements and its limitations.
%What we have learned, what went well and what that could have been done better.
Additionally, we talk about social dynamics within the group and with involved parties.

\section{The product}

We were happy with our final product and satisfied with the amount of work we could actually accomplish.
This was thanks to having agreed with the customer on the scope of the problem and by having re-used a
lot of open-source code. We were definitely happy with the project's task which led us to study trending topics
like eHealth and self-quantification. Although we had no previous experiences in designing and developing a web API,
we found it interesting and enjoyable. This resulted in a proactive participation by the group
in the improvement and proposal of requirements, a process which in turn made the project even more interesting
for us to work on.

\subsection{Results}

This project involved designing and developing a number of interoperable systems consisting of:
\begin{itemize}
\item an integration platform
\item a web front-end
\item two Android applications
\item a web application using JSP
\end{itemize}
to analyze the feasibility for a national scale eHealth project to leverage the wide availability of devices
like smartphones and other proprietary solutions like HealthVault.

We have designed and implemented an integration platform as a Spring based web application.
This application exposes a JSON based API through which is possible to exchange data models
representing health measurements such as heart rate and weight.
The web application is deployed using Apache Tomcat and implements persistence using a MySQL database.

%\clearpage
The remaining systems have been developed to demonstrate:
\begin{enumerate}[a.]
\item the functionality of the integration platform by using its API to store and retrieve health measurements
\item the feasibility of integrating with already existing third-party solutions such as Microsoft's HealthVault
\item the difficulty for third-party developers to use the API to make their applications interoperable with the integration platform
\end{enumerate}

Our results outline not only the possibility to leverage handheld devices, which we have demonstrated
using the \textit{Heart rate} application, but also the ease with which third-party developers can make
their application interoperable with our implementation of the integration platform.
In fact, this has been accomplished in both Android applications with less than 100 LOC.
Additionally, our \textit{Weight} application and \textit{HealthVault integration service} have shown how
it is possible to integrate information from other solution provides as well.

Our architectural design choices make the platform well suited to be scaled up and
complemented with security practices. The integration platform in fact, consists
of a web application and a database which can be deployed separately, reducing the amount of resources
needed on a single machine. The same applies to the front-end as well because it interacts with the
integration platform using its API just like any other application and therefore does not rely on being
deployed on the same machine. Furthermore, because we are not employing a classical SOAP client-server
architecture but a more lean, REST-like one, the implementation can be considered lightweight in term of
resources and easier for application developers to interact with.

\subsection{Requirements}

We managed to achieve all requirements and collaborated actively with the customer in the proposal
of new features. However, our product is still only a demonstrative prototype and there is
much room for improvement. Table \ref{table:fulfill-req} presents a complete overview the status of requirements.

\begin{table}[h]
\begin{center}
\begin{tabular}{ | c | c | c | c | c | }
  \hline
  \textbf{ID} & \textbf{Category} &\textbf{Complexity} & \textbf{Priority} & \textbf{Status}\\
  \hline\noalign{\smallskip}\noalign{\smallskip}\hline
  FIP1	& NIPEN				& Med	& High & Completed \\
  FIP2	& NIPEN				& Med	& High & Completed \\
  FIP3	& NIPEN				& Med	& High & Completed \\
  FPI4	& NIPEN				& Med	& High & Completed \\
  FW1   & Front-end			& Med	& High & Completed \\
  FW2   & Front-end			& Low	& Low  & Completed \\
  FHR1	& Heart rate app	& High	& High & Completed \\
  FHR2	& Heart rate app	& Low	& High & Completed \\
  FHR3	& Heart rate app	& Med	& High & Completed \\
  FHV1	& Weighter app		& Med	& High & Completed \\
  FHV2	& Weighter app		& Low	& Med  & Completed \\
  FHV3	& Weighter app		& Low	& High & Completed \\

  FHIS1	& HV integration service	& Med   & High & Completed \\
  FHIS2	& HV integration service	& High  & High & Completed \\
  FHIS3	& HV integration service	& Low	& Med  & Completed \\

  NF1 & Documentation		& High 	& High	& Completed \\
  NF2 & Documentation		& Med 	& High	& Completed \\
  NF3 & Open-source			& Low 	& High	& Completed \\
  NF4 & Interoperability	& Med	& High	& Completed \\
  NF5 & Interoperability	& High	& High	& Completed \\
  NF6 & Accessibility		& Low	& Low	& Completed \\

  \hline
\end{tabular}
\end{center}
\caption{Fulfillment of requirements}
\label{table:fulfill-req}
\end{table}


\subsection{Product limitations}
\label{subsec:limitations}

As every software product, our product has some limitations which should be discussed.
What we present here is a list of the most remarkable limitations of our product.
%that we consider should also be the focus for further development.

\textbf{API design}\newline
At the moment the API only supports two types of data models: weight and heart rate measurements.
The design of an API's data models will have a key role in its success and adoption by
third-parties. What we proposed are two very simple models which include minimal information.
Furthermore, the API is not implementing the basic features of persistent storage (CRUD)
which are necessary for this kind of solution. Only create and read operations are supported,
whereas update and delete are totally missing.

\textbf{Security}\newline
No part of the system implements any form of security.
This was done to reduce the complexity of the system due to time constraints and size of the group.
The customer agreed that security was not a requirement of the system, however in a real-world scenario
appropriate security practices must be adopted.
Especially in projects of this type and scale, security will be a critical requirement
to take into consideration when designing the system and the API as well.

\textbf{Multi-user support}\newline
Some parts of the system such as the API data models, internal data structures and even
database tables do have a basic support for more than one user. However, because no authentication
mechanism is implemented, all actions are effectively carried out by a single user.
Naturally, in a real-world scenario that would not be the case and multi-user support will
be necessary.

\textbf{Deployment}\newline
At the moment both NIPEN, the database and the front-end are deployed together on single server machine.
It would be desirable to deploy these on different machines as that would reduce
the amount of resources needed for a single machine especially considering a large userbase.



\section{Development process}
All in all, we were satisfied with our choice of Scrum and how we adapted
it to our specific needs (see \ref{subsec:devprocess}).
Two weeks sprints proved to be just the right duration to achieve
a high-level goal. Additionally, because there were a total of six sprints,
we had enough chances to adjust our plan to accommodate for changes in requirements.
The overhead introduced by Scrum was negligible and this has proven valuable
given also our resource constraints. There were however some minor issues with our process.

\textbf{Time tracking}\newline
Sometimes, although we had tools to conveniently track the the time spent on the project,
not everybody would do that in due time. This has hindered the effectiveness of burndown charts
to provide meaningful information on the project's progress and could have led to planning issues.
It would have helped to get used to report our times at the end of each day.

\textbf{Effort estimates}\newline
It should be noted that we didn't use any empirical method to estimate the amount of effort
required to complete an activity: the scrum master would simply make a guess.
Especially in the beginning when most of the work was field study and the project
was in the early stages, it was difficult to produce meaningful estimates.
However, using an online tool for scrum we managed to keep an eye on previous
estimates and adjust them based on the amount of effort actually required to complete the tasks.
Furthermore as we became familiar with involved technologies and our own productivity,
the estimates became more and more accurate.

\textbf{Planning}\newline
We think that while our planning was good, at some moments our plan maybe have been not.
Especially in the beginning our plan was not very well structured and this was reflected
by a somewhat overcrowded sprint backlog and less effective means to get an overview
of real the progress of the project. However, our planning became increasingly
methodical, thanks also to the fact that we were using an iterative approach which
left a lot of room for improvement.%, and this resulted in?

\section{Social dynamics}
This section discusses social dynamics during the project, both internally and with
involved parties.

\subsection{Group dynamics}
\label{sec:group}

Group dynamics were definitely a critical aspect of this project.
It would be safe to say that due to the nature of the project itself, its success was based primarily on group dynamics.
We were, in fact, a group of people working to deliver a product to another group of people.
It is easy to convince ourselves that people management in a software process is an essential task for a successful product.
Any successful software project relies on successful group dynamics because no project can be
accomplished by a single man.% We understand that people management will be a critical aspect of our process.
%This encompasses a set of practices, precautions and knowledge which we have taken into consideration and tried our
%best to make our own.

\textbf{Motivate}\newline
Team's commitment will greatly improve many variables in a software project.
People are inevitably influenced by other people around them, especially when they work together.
If any member has a low commitment to the project his attitude will influence negatively other team members.

%A good project manager will motivate the team to work well and gratify members who do so.
%It is a duty of the project manager to instill a positive and enthusiastic attitude in the team.

%To achieve this task, it is important to identify personality types among a group so that
%motivation can be more effective. These can be divided into three categories:
%  \begin{itemize}
%  \item Self-oriented: the work has the purpose to achieve personal interestes.
%  \item Task-oriented: the motivation is related to the task itself.
%  \item Interaction-oriented: the motivation is related to social interaction.
%  \end{itemize}

%It is desiderable that within a group such personality types are balanced.

In our case, the scrum master tried motivating the group by rewarding good jobs
and outlining the importance of the project itself as a chance to acquire knowledge,
to get a good grade (academic achievement) and to do something that stimulated further
research and work in the field.

Feedback from the supervisor and the customer has been constructive and
contributed to further motivate the group.

\textbf{Cooperation}\newline
People should be encouraged to work together, share ideas and thoughts.
We should avoid having one team member whose role is predominant on others'.

During meetings, both internal and external, we had one team member
act as a meeting leader, guiding the customer, supervisor or other team members through
the meeting agenda. However, the meeting leader made sure to involve all team members
in the conversation, asking their opinion on the current topic being discussed.
When the supervisor or the customer asked questions, he made sure not to be the only one
answering by inviting other team members to do so.

\textbf{Size matters}\newline
The fact that the group consisted of three members definitely played a factor in both group dynamics
and project management. Though many may see it as a weakness we believe in our case it was not,
or rather it proved more a strength than a weakness.
In software development, just like in any other engineering practice assigning many people
at a task is unlikely to produce better results than a sufficiently small but motivated group.
%Just like it takes one month to ferment wine, no matter how many people are involved.
Although having a small team means that flexibility in group composition is absent and
the technical skill pool is reduced it is important to realize that a small team
can be more easily motivated and achieve a higher degree of cooperation.
To do so, care should be taken into not letting any member of the group become
a 'black sheep'. Team members should respect each others and their respective work,
nobody should feel less important or cut out of the decision making process.

\textbf{Distance matters}\newline
During the project one team member moved to Oslo and got a full-time job.
This important event required some adjustment to our project management strategy,
especially regarding planning. We thought that the easiest way for him to continue
to contribute to the project was to concentrate on documentation.
This has proved to be generally a good idea although it would have helped
to be more specific about what parts of the documentation he should have focused on.


\subsection{Customer relations}

%\textbf{Relations}\newline
We were definitely pleased with our relations with the customer.
Since our first meeting he was able to explain what would have been the purpose of this project and why
it was relevant for him. He was able to capture our interest and left the task 'open' enough to be stimulating for us.
We collaborated in thinking about features that could have made the product more valuable for him.
The customer was a nice person, smiling and available. He gratified us when he thought we were doing a good job and
that has motivated us to keep it up and work hard. We were glad to be able to work with him.

\textbf{Communication}\newline
Since our customer was located in Oslo, we didn't have many chances to meet in person,
in fact that happened only once. Nevertheless we did have meetings on Skype on a regular weekly basis
with some exceptions due to his work schedule or personal issues.
This wasn't a problem and we were able to manage our work pretty well and autonomously
even those weeks we didn't hear from him.

\subsection{Supervisor relations}

We are happy with our relations with the supervisor.
He made his role in the project clear to us from the beginning so that we knew
what we could count on him for and what not.
We reported him the status of our project weekly and he would always have some feedback
on our progress and also make sure to have a proper understanding of our plans.
His feedback was precious to our project and certainly helped us not to miss any important details.

\textbf{Communication}\newline
Our communication with the supervisor was regular, on a weekly basis.
We had meetings in person on Mondays, agendas being sent one day earlier by mail.
Probably that didn't leave much time to the supervisor to review our documentation.
Additionally, in the beginning we didn't understand very well what kind of documents
we needed to provide him.
However, we feel that these have hardly been problems as we always gave him a
good amount of information on our process and the project's progress and we eventually
become methodical in producing the documentation he needed.

