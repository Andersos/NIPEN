
\chapter{Sprint 2}
\label{Sprint2}
\lhead{Chapter 8. \emph{Sprint 2}}

\section{Goal(s)}
Having just demonstrated the product to the customer during the previous sprint, the main goal
for this iteration was to write the report in view of the second milestone deadline for mid-term delivery (M2).
Additionally we had to assess the feasibility of interoperability of our system with HealthVault
and begin to add support for weight measurements to the API.

\section{Duration}
The duration of the sprint was the following:
\begin{itemize}
\item Start: September, 23rd
\item End: October, 6th
\end{itemize}

\section{Planning}

Most of the work planned for this sprint was documentation writing.
This included writing the report itself but also reviewing other documents such as the
templates, notes and agendas. Although one member of the team would be working remotely
from another city, we thought it would be easy for him to collaborate in writing the documentation
so we did not prepare a separate work plan for him.
We did, though, schedule an internal meeting with him to be held weekly on Mondays
to keep in touch and share information about the project's progress.

Because we felt confident that interoperability with HealthVault was possible
we planned to work on an initial design of a second prototype application to demonstrate it.

%work on the development of a second Android application which would connect
%to HealtVault in order to retrieve data about user's weight and forward it to our system.
%For this task we planned to use HealthVault's SDK for Android.

Furthermore, we planned to continue with system development and add support for weight
measurements to the API.

%We wanted to begin some preliminary work toward the second prototype of the system which would
%feature HealthVault interoperability. Our plan was to develop an Android application 

\section{Backlog}

See below the sprint backlog.
\begin{enumerate}[1.]
	\item \textbf{Project management} included:
	\begin{itemize}
		\item \textbf{Weekly startup meeting}
		\item \textbf{Weekly meetings}
			meetings with both the customer and the supervisor.
			%*During week 39.we were unable to contact the customer. This was not a problem*
		\item \textbf{Meeting notes}:
			taking notes during meetings, reviewing of the notes.
		\item \textbf{Status reports}:
			for both week 39 and 40
		\item \textbf{Review templates}:
			to be adopted for status reports and other documents.
		\item \textbf{Risk analysis}:
			updated on a weekly basis, so twice per sprint.
			%The risk analisys was submitted to the supervisor and the customer.
		\item \textbf{Planning for the next iteration}:
			prepare a plan for the next iteration
			%which would be illustrated and agreed upon on next iteration's startup meeting.
	\end{itemize}
	\item \textbf{Report work}\newline
		start writing the report, focus on necessary chapters for the mid-term delivery.
		%This tasks was definitely the biggest of the sprint and accounted for the majority of the work.
		%We focused on writing the abstract and first chapters required for the mid-term delivery.
	\item \textbf{HealthVault studies}\newline
		perform further studies in order to assess the feasibility of developing a prototype
		application that implemented interoperability with Microsoft's platform.
	\item \textbf{System development}\newline
		preliminary development on the backend to accomodate the new weight API.
	\item \textbf{Application development}\newline
		initial design and implementation of a prototype application using the weight API.
		The application needs to connect to HealthVault and retrieve some weight measurements
		and then send them to the integration platform.
\end{enumerate}


\section{Results and feedback}

We wrote a table of contents for the report and parts of the initial chapters as well.
The table of contents was submitted to the supervisor in order to make sure we weren't leaving out
any important chapter in our report. The team member who had moved to Oslo was able to contribute
significantly to the project by writing the report.

Studies on HealtVault went smoothly. Our idea for the second prototype application was
to write another Android application which would fetch some weight measurements from HealthVault
and send them to our integration platform. We presented this idea to the customer and he agreed on it.

Development has proceeded smoothly. This was also thanks to HealthVault's SDK which greatly
reduced the amount of code we needed to write ourselves from scratch.
We used one of the examples provided with the SDK which featured out-of-the-box interoperability
with HealthVault and added the functionality required to make it interoperable with our system as well.

System development also proceeded smoothly and we added support for the weight data to the backend.
The frontend and the database needed little work to accomodate these additions, so most of the work done on
them was minor tweakings.

\section{Evaluation}

Results for this sprint were positive. We were pleased with how the report was shaping up and
received good feedback on our structure from the supervisor.

Our plan to let our colleague in Oslo work on the documentation worked out pretty well
and he was able to work indipendently and effeciently. Although he had moved to Oslo,
he didn't start to work right away so he a lot of time to dedicate to the project.

Since we had told the customer that we would be working primarily on the report for this sprint,
he didn't put any pressure on us to develop new features.
We were happy that our project was well in schedule with our plan so far.

