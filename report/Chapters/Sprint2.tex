

\chapter{Sprint 2}
\label{Sprint0}
\lhead{Chapter 8. \emph{Sprint 2}}

\section{Goal(s)}
Having just demonstrated the product to the customer during the previous sprint, the main goal
for this iteration was to write the report in view of the second milestone deadline for mid-term delivery.
Additionally we planned to assess the feasibility of interoperability of our system with HealthVault.

\section{Planning}

Most of the work planned for this sprint was documentation writing.
This included writing the report itself but also reviewing other documents such as the
templates, report, notes and agendas. Although one member of the team would be working remotely
from another city, we thought it would be easy for him to collaborate in writing the documentation
so we did not make any special planning for him.
We wanted to begin some preliminary work toward the second prototype of the system which would
feature HealthVault interoperability. Our plan was to develop an Android application which would connect
to HealtVault in order to retrieve data about user's weight and forward it to our system. For this task
we planned to use HealthVault's SDK for Android.

\section{Duration}
\begin{itemize}
\item Start: September, 23rd
\item End: October, 6th
\end{itemize}

\section{Backlog}

See below the sprint backlog.
\begin{itemize}
	\item \textbf{Project management}\newline
	This included:
	\begin{itemize}
		\item \textbf{Weekly startup meeting}:
		\item \textbf{Meeting notes}:
			taking notes during meetings, reviewing of the notes.
		\item \textbf{Status reports}:
			for both week 39 and 40
		\item \textbf{Review templates}:
			we reviewed some templates to be adopted for status reports and other documents.
		\item \textbf{Risk analysis}:
			updated on a weekly basis, so twice per sprint.
			The risk analisys was submitted to the supervisor and the customer.
		\item \textbf{Planning for the next iteration}:
			the project manager prepared a plan for the next iteration
			which would be illustrated and agreed upon on next iteration's startup meeting.
	\end{itemize}
	\item \textbf{Weekly meetings}
		included meetings with both the customer and the supervisor.
		*During week 39.we were unable to contact the customer. This was not a problem*
	\item \textbf{Report work}:
		we started writing the report.
		This tasks was definitely the biggest of the sprint and accounted for the majority of the work.
		We focused on writing the abstract and first chapters required for the mid-term delivery.
	\item \textbf{HealthVault studies}:
		we performed further studies in order to assess the feasibility of developing a prototype
		application that implemented interoperability with Microsoft's HealthVault platform.
	\item \textbf{System development}:
		this included some preliminary work on the backend to accomodate for the new weight API.
	\item \textbf{Application development}:
		Initial design and implementation of an Android application using the weight API.

\end{itemize}


\section{Results and feedback}

We managed to write a table of contents for the report and began writing the initial chapters as well.
The team member who had moved to Oslo was able to contribute to the project writing the report.
Also, studies on HealtVault went smoothly and we had a good start at developing the Android application
especially. In fact, using HealthVault's SDK greatly reduced the amount of code we needed to write ourselves
from scratch.
System development also proceeded smoothly and we added support for the weight data to the backend.
The frontend needed little work to accomodate these additions, so most of the work done on it
was minor tweakings.

\section{Evaluation}

Results for this sprint were positive. We were pleased with how the report was shaping up and
received good feedback on our structure from the supervisor.
%We were also happy with the amount of effort put into report writing by the team member in Oslo.



