

\chapter{Sprint 1}
\label{Sprint0}
\lhead{Chapter 7. \emph{Sprint 1}}

\section{Goal(s)}
This sprint's goal was the completition of an initial prototype of the system whose functionality
had to be demonstrated to the customer; this corresponded to project milestone M1.
The purpose of such prototype was to act as a proof-of-concept and give the customer
a chance to express some feedback upon which we hoped to start a discussion
on possible improvements and new features.

\section{Planning}
In order to achieve the goal for this sprint we planned to accomplish an initial design
and implementation of every part of the system including:
a) the frontend b) the backend and database c) the Android application to measure heart rate.
We planned some additional studies on relevant technologies for the first week of the sprint
and begin writing some parts of the report.

\section{Duration}
The duration of the sprint was the following:
\begin{itemize}
\item Start: September, 9th
\item Milestone M1 (first system prototype): September, 20th
\item End: September, 22nd
\end{itemize}

\section{Backlog}

See below the sprint backlog.
\begin{itemize}
	\item \textbf{M1 First system prototype}
	\item \textbf{Project management}\newline
	This included:
	\begin{itemize}
		\item \textbf{Weekly startup meeting}: 
		\item \textbf{Meeting notes}:
			taking notes during meetings, reviewing of the notes.
		\item \textbf{Status reports}:
			for both week 37 and 38
		\item \textbf{Risk analysis}:
			updated on a weekly basis, so twice per sprint.
			The risk analisys was submitted to the supervisor and the customer.
		\item \textbf{Planning for the next iteration}:
			the project manager prepared a plan for the next iteration
			which would be illustrated and agreed upon on next iteration's startup meeting.
	\end{itemize}
	\item \textbf{Weekly meetings}
		includes meetings with both the customer and the supervisor.
		The meeting with the customer was held on Skype.
	\item \textbf{Additional pre-studies}
		Continued studies on relevant technlogies such as:
	\begin{itemize}
		\item \textbf{HealthVault}: Microsoft's online health platform.
		\item \textbf{Apache Camel}: a routing engine for enterprise integration patterns.
		\item \textbf{Javascript libraries for charts}: to be used in the frontend.
	\end{itemize}
	\item \textbf{System development}
		Initial design and implementation. For this sprint, this accounted for:
	\begin{itemize}
		\item \textbf{Backend development}:
			Development of Spring controllers for API endpoints. Database (DAO) coding to
			implement data persistence.
		\item \textbf{Frontend development}:
			Coding of the frontend. This included setting up an HTML page which used
			AJAX to perform API calls and JS to show the data retrieved using a bar chart.
		\item \textbf{Deployment}:
			Deployment of both backend and frontend using a servlet container (Tomcat).
	\end{itemize}
	\item \textbf{Heart rate application}:
		basic implementation of the Heart rate application. The application should be able to acquire
		the user's heart reate and send perform an API call to store the data on the backend.
	\item \textbf{Database development}:
		We chose a suitable database to use for implementing persistency on the backend.
		We opted for MySQL due to the familiarity we had with it. We then deployed it on the server machine
		and added a first table so that we could test the functionality of the backend.
	\item \textbf{Testing}:
		Perform unit and integration testing for the heart rate application and the backend.
\end{itemize}


%% i actually dont think having a table for this is a good idea. tables can't have much information
%% also it would require a lot of tweaking to get the estimated/actual times to look reasonable.
%% maybe a textual description would be better so we can omit some details in favor of others.
\iffalse
\begin{table}
\begin{tabular}{ | l | l | l | l | }
 \hline
  Story ID & Description & Size & Assignee \\
  \hline\noalign{\smallskip}\noalign{\smallskip}\hline
  33 & Project Management			& 8	& Emanuele  \\
  12 & M1 First System prototype	& 0 & All		\\
  45 & Weekly meetings (week 37)	& 6 & All		\\
  42 & Additional pre-studies		& 5 & Emanuele	\\
  \hline
\end{tabular}
\caption{}
\label{}
\end{table}
\fi

\section{Testing}

\section{Results and feedback}

Around the end of this sprint, on September 20th, we demonstrated a prototype of system to
the client which included: a) the frontend, b) the backend, c) an Android application to measure
heart rate using the phone's camera.

The backend supported web API calls for storing and retrieving heart rate measurements.
Both the Android application and the frontend used these API calls for sending and retrieving
such measurements respectively. The measurements were ultimately shown to the user on a
webpage (the frontend) using a bar chart.

The customer was pleased with the results, stating that they were above his expectations.

We discussed about which features he would like to see implemented next in the product
and we agreed to prioritise interoperability with HealthVault instead of Withings.

The reason behind this decision was the fact that interoperability with HealthVault would have
enabled the product to use third-party devices supported by HealthVault to gather health data.

Since the number of devices supported by HealthVault is substantial, this was deemed a
desiderable feature for the product.

The customer asked if the amounts of resources we had at disposal to actually implement
such interoperability was sufficient and inquired about the general viability of such approach.

We expressed our confidence about the feasibility of such approach and that the resources
at hand were sufficient. Nevertheless, we set a deadline (10 days) to assess such
feasibility in order not to delay the general progress of the product.

\textbf{Notes}: before the end of the sprint one team member had moved permanently to Oslo.

we didn't manage to begin writing the report.

\section{Evaluation}

The sprint was successful as we managed to achieve the main goal of the sprint (Milestone M1).
We were pleased by the positive feedback received from the customer and felt motivated to keep up the good work.
Furthermore, having proactively partecipated in the improvement/proposal of product requirements
together with the customer made the whole team look forward to implement these new features and improve the product.

Although we didn't manage write anything in the report we still had