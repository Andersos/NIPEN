
\chapter{Testing}
\label{Testing}
\lhead{Chapter 13. \emph{Testing}}
\nocite{SoftwareTesting}

%\section{Testing types}

%In this section we are going to give a brief explanation of the different types of testing.

%\subsection{Black box testing}

%This type of testing sees the system as a black box.
%What this means is that we are only giving the system inputs and viewing how the system reacts on those inputs, without %concerning how the system works internally. 
%We are using black box testing when we are performing system testing and acceptance testing.

%\subsection{White box testing}

%When doing white box testing we are testing internal structures of the system.
%We are using white box testing when performing unit tests.

%\subsection{Gray box testing}

%Gray box testing is a combination of black box and white box testing.

%\section{Testing methodologies}

In this chapter we are going to elaborate on the testing methodologies used and how we performed the different tests.
Do to time restrictions and the fact that we only were three, we only preformed manual testing.
The results of the functional tests performed, can be viewed in appendix \ref{AppendixD}.

\section{Unit testing}

In unit testing a test is performed for each individual part of the source code.
This is the lowest level of testing, and often a tool is used for this process like JUnit.
In our case we unfortunately didn't have time to implement JUnit tests, so most of this tests were conducted manually.
This was usually performed by testing a method after it was created, and then test if it worked as it should.

\section{Integration testing}

Since our system consists of multiple applications, integration testing is important.
Here we are testing if all parts of the system are working together.
Each application in our system has its own interface that it uses to communicate with other parts of the system.
For example the NIPEN server application uses an API as its interface for sending and receiving data.
In integration testing it is important to test if each interface works as it should.

The following tasks were performed during integration testing:

\textbf{Database}
\begin{enumerate}
\item Create a weight model and store it in the database, then retrieve it and compare all the values.
\item Create a heart rate model and store it in the database, then retrieve it and compare all the values.
\end{enumerate}

\textbf{NIPEN}
\begin{enumerate}
\item Create some weight models and retrieve them through an API call. Check if all the data is correct.
\item Create some heart rate models and retrieve them through an API call. Check if all the data is correct.
\item Create a weight JSON string and POST it to NIPEN. Construct a weight model based on the JSON string on the back-end. Check if the values are correct.
\item Create a heart rate JSON string and POST it to NIPEN. Construct a heart rate model based on the JSON string on the back-end. Check if the values are correct.
\end{enumerate}

\textbf{Front-end}
\begin{enumerate}
\item Perform an AJAX call to the server and receive heart rate values. Display the heart rate values on the front-end and check if they are the same as in the received JSON string.
\item Perform an AJAX call to the server and receive weight values. Display the weight values on the front-end and check if they are the same as in the received JSON string.
\end{enumerate}

\textbf{Heart Rate Application}
\begin{enumerate}
\item Construct a heart rate JSON string and send it to NIPEN. Check on the server application if the received JSON message has the right format and contains the right values.
\end{enumerate}

\textbf{Weight Application}
\begin{enumerate}
\item Construct a weight JSON string and send it to NIPEN. Check on the server application if the received JSON message has the right format and contains the right values.
\end{enumerate}

\textbf{HealthVault Integration Service}
\begin{enumerate}
\item Construct a weight JSON string and send it to NIPEN. Check on the server application if the received JSON message has the right format and contains the right values.
\end{enumerate}

\section{System testing}

In system testing we are testing if all the functionalities of the whole integrated system are working as they should.
When performing this test we connected all the systems that were dependent on each other and tested them together.
This consisted of three main tests:

\begin{enumerate}
\item Pushing data with the heart rate application
\item Pushing data with the weight application
\item Pushing data with the HealthVault integration service
\end{enumerate}

By performing the tasks mentioned above, we were able to test all parts of the system.
This includes the REST endpoints of NIPEN, the database, the front-end and each of the applications that were used to push data into the system.
When pushing a value into NIPEN it is first received at the REST endpoints of NIPEN, then it is sent and stored in the database.
After that the front-end asks NIPEN to receive new data.
This data is again gathered from the database and sent to the front-end, which again will display the data for the user.
Hence all the main parts of the system are being tested by performing the mentioned tasks.

\section{Acceptance testing}

Acceptance testing is a test to determine if all the requirements of the application are met.
These tests were performed by showing a demonstration of the application to the customer.
We were able to demonstrate the system two times.
During the first demonstration we had implemented everything that was needed to store and display heart rate measurements.
Hence, the acceptance test was based on the following functionalities:

\begin{itemize}
\item NIPEN REST end points for retrieving and receiving heart rate data.
\item Database for storing heart rate data.
\item Front-end for displaying heart rate data.
\item Heart rate application for creating and sending heart rate data.
\end{itemize}

We demonstrated the functionalities to the customer, by measuring and sending multiple heart rate measurements through the heart rate application.
The values sent from the Android device were successfully displayed on the front-end after they were sent.
Thus this acceptance test was successful since the customer was pleased with the results.

During our second demonstration we were able to demonstrate the part of the system concerning weight data.
This includes the following parts:

\begin{itemize}
\item NIPEN REST end points for retrieving and receiving weight data.
\item Database for storing weight data.
\item Front-end for displaying weight data.
\item Weight application for creating and sending weigh data.
\item HealthVault integration service for creating and sending weight data.
\end{itemize}

The test was performed by first demonstrating the HealthVault integration service that we implemented.
We started the polling service and let the customer enter some values into HealthVault.
The values were sent and stored in our system and displayed on the front-end.
Then we demonstrated the weight application for our customer by entering a value into the application, and sending it into both HealthVault and NIPEN.
The values were successfully transmitted and stored in both systems.
This acceptance test also was successful, since the customer was satisfied with this demonstration and the requirements were met.