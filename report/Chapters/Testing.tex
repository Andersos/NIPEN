
\chapter{Testing}

\label{Testing}

\lhead{Chapter 12. \emph{Testing}}

\section{Testing methodologies}

\section{Tools}

\section{Functional tests}

\begin{table}
\begin{center}
\begin{tabular}{ | l | p{10cm} | }
	\hline
	\textbf{Test}	&	\textbf{ID 1} \\
	\hline\noalign{\smallskip}\noalign{\smallskip}\hline
	Name				& Heart rate REST (read) \\
	Requirement			& \hyperref[table:reqip]{FIP1} \\
	Description			& Test the REST endpoint for receiving heart rate data models \\
	Preconditions		& 	\par The IP is deployed and running on a server machine 
							\par The test is run on the server machine or alternatively
							one that has access to the server and whose address is replaced to the
							string \verb|localhost| in the test.
							\par The machine on which the test is run has the \verb|curl| program installed. \\
	Steps 				&	1. Run the following:
							\begin{verbatim}
							curl localhost:8080/nipen/human/api/heart_rates
							\end{verbatim}
							\\
	Postconditions		& A JSON valid, comma separated list of heart rate data models consistent with 
							entries in the databased hosted on the server. \\
	Results				& - \\
	Comments			& - \\
	Status				& OK or FAIL \\
	Tester				& Person \\
	Date				& dd-mm-2013 \\
	\hline
\end{tabular}
\end{center}
\end{table}

\begin{table}
\begin{center}
\begin{tabular}{ | l | p{10cm} | }
	\hline
	\textbf{Test}	&	\textbf{ID 2} \\
	\hline\noalign{\smallskip}\noalign{\smallskip}\hline
	Name				& Weight REST (read) \\
	Requirement			& \hyperref[table:reqip]{FIP2} \\
	Description			& Test the REST endpoint for receiving weight data models \\
	Preconditions		&	\par The IP is deployed and running on a server machine
							\par The test is run on the server machine or alternatively
							one that has access to the server and whose address is replaced to the
							string \verb|localhost| in the test.
							\par The machine on which the test is run has the \verb|curl| program installed. \\
	Steps 				&	1. Run the following:
							\begin{verbatim}
							curl localhost:8080/nipen/human/api/weights
							\end{verbatim}
							\\
	Postconditions		& A JSON valid, comma separated list of heart rate data models consistent with 
							entries in the database hosted on the server. \\
	Results				& - \\
	Comments			& - \\
	Status				& OK or FAIL \\
	Tester				& Person \\
	Date				& dd-mm-2013 \\
	\hline
\end{tabular}
\end{center}
\end{table}

\begin{table}
\begin{center}
\begin{tabular}{ | l | p{10cm} | }
	\hline
	\textbf{Test}	&	\textbf{ID 3} \\
	\hline\noalign{\smallskip}\noalign{\smallskip}\hline
	Name				& Heart rate REST (write) \\
	Requirement			& \hyperref[table:reqip]{FIP3} \\
	Description			& Test the REST endpoint for requesting heart rate data models \\
	Preconditions		&	\par The IP is deployed and running on a server machine
							\par The test is run on the server machine or alternatively
							one that has access to the server and whose address is replaced to the
							string \verb|localhost| in the test.
							\par The machine on which the test is run has the \verb|curl| program installed.
							\par At least one heart rate measurement has been stored on the IP \\
	Steps 				&	1. Run the following \begin{verbatim}
							curl -X POST -H "Content-Type: application/json"
							-d {"userId":0,"value":60,"timestamp":"11-10-2013","unit":
							"bpm"}' localhost:8080/nipen/human/api/heart_rates
							\end{verbatim} \\
	Postconditions		& A database entry coherent with the JSON data submitted is created on the server. \\
	Results				& -- \\
	Comments			& - \\
	Status				& OK or FAIL \\
	Tester				& Person \\
	Date				& dd-mm-2013 \\
	\hline
\end{tabular}
\end{center}
\end{table}


\begin{table}
\begin{center}
\begin{tabular}{ | l | p{10cm} | }
	\hline
	\textbf{Test}	&	\textbf{ID 4} \\
	\hline\noalign{\smallskip}\noalign{\smallskip}\hline
	Name				& Weight REST (write) \\
	Requirement			& \hyperref[table:reqip]{FIP4} \\
	Description			& Test the REST endpoint for requeting weight data models \\
	Preconditions		&	\par The IP is deployed and running on a server machine
							\par The test is run on the server machine or alternatively
							one that has access to the server and whose address is replaced to the
							string \verb|localhost| in the test.
							\par The machine on which the test is run has the \verb|curl| program installed.
							\par At least one weight measurement has been stored on the IP \\
	Steps 				&	1. Run the following \begin{verbatim}
							curl -X POST -H "Content-Type: application/json"
							-d {"userId":0,"value":60,"timestamp":"11-10-2013","unit":
							"kg"}' localhost:8080/nipen/human/api/weight
							\end{verbatim} \\
	Postconditions		& A database entry coherent with the JSON data submitted is created on the server. \\
	Results				& -- \\
	Comments			& - \\
	Status				& OK or FAIL \\
	Tester				& Person \\
	Date				& dd-mm-2013 \\
	\hline
\end{tabular}
\end{center}
\end{table}


\begin{table}
\begin{center}
\begin{tabular}{ | l | p{10cm} | }
	\hline
	\textbf{Test}	&	\textbf{ID 5} \\
	\hline\noalign{\smallskip}\noalign{\smallskip}\hline
	Name				& Heart rate measurement \\
	Requirement			& \hyperref[table:reqheartrate]{FHR1} \\
	Description			& Test the heart rate measurement functionality \\
	Preconditions		& The application has started \\
	Steps 				&	\par 1. User holds his finger on the camera applying a slight pressure \\
	Postconditions		&	\par 1. The icon on the left side of the screen is blinking 
							\par 2. A measurement appears after 4 seconds at most \\
	Results				& -- \\
	Comments			&	The measurement is expected to be rough.
							Any value between 60 and 100 is okay as long as it actually varies slightly based
							on the tester's perceived heart rate.  \\
	Status				& OK or FAIL \\
	Tester				& Person \\
	Date				& dd-mm-2013 \\
	\hline
\end{tabular}
\end{center}
\end{table}


\begin{table}
\begin{center}
\begin{tabular}{ | l | p{10cm} | }
	\hline
	\textbf{Test}	&	\textbf{ID 6} \\
	\hline\noalign{\smallskip}\noalign{\smallskip}\hline
	Name				& Login \\
	Requirement			& FR1 \\
	Description			& Test the login functionality \\
	Preconditions		& The application has started \\
	Steps 				&	\par 1. 
							\par 2. 
							\par 3. \\
	Postconditions		& The login screen disappear and the inbox appear \\
	Results				& -- \\
	Comments			& -- \\
	Status				& OK or FAIL \\
	Tester				& Person \\
	Date				& dd-mm-2013 \\
	\hline
\end{tabular}
\end{center}
\end{table}


\begin{table}
\begin{center}
\begin{tabular}{ | l | p{10cm} | }
	\hline
	\textbf{Test}	&	\textbf{ID 7} \\
	\hline\noalign{\smallskip}\noalign{\smallskip}\hline
	Name				& Heart rate measurement \\
	Requirement			& \hyperref[table:reqheartrate]{FHR3} \\
	Description			& Test the heart rate application interoperability \\
	Preconditions		&	\par 1. The application has started
							\par 2. A heart rate measurement has been acquired \\
	Steps 				&	\par 1. User presses \textbf{Send} button \\
	Postconditions		&	\par 1. The application shows a toast
							\par 2. The measurement is acquired by the Integration Platform \\
	Results				& -- \\
	Comments			& -- \\
	Status				& OK or FAIL \\
	Tester				& Person \\
	Date				& dd-mm-2013 \\
	\hline
\end{tabular}
\end{center}
\end{table}

\begin{table}
\begin{center}
\begin{tabular}{ | l | p{10cm} | }
	\hline
	\textbf{Test}	&	\textbf{ID 8} \\
	\hline\noalign{\smallskip}\noalign{\smallskip}\hline
	Name				& HealthVault data fetching \\
	Requirement			& \hyperref[table:reqweight]{FHV1} \\
	Description			& Test HealthVault connectivity \\
	Preconditions		&	\par 1. The application has started
							\par 2. The user has authenticated to HealthVault \\
	Steps 				&	\par 1. User \\
	Postconditions		&	\par 1. 
							\par 2. \\
	Results				& -- \\
	Comments			& -- \\
	Status				& OK or FAIL\\
	Tester				& Person \\
	Date				& dd-mm-2013 \\
	\hline
\end{tabular}
\end{center}
\end{table}