\chapter{Requirements specification}

\label{ch:requirements}
\lhead{Chapter 4. \emph{Requirements specification}}

This chapter describes the requirements for the product. These include both functional and non-functional requirements.
Functional requirements can be found in section \ref{section:functionalreq},
non-functional requirements in section \ref{section:nonfunctionalreq}.

Throughout this chapter both requirement's priority and complexity have a textual description which can be
\begin{itemize}
\item High
\item Medium (abbr. Med)
\item Low
\end{itemize}

The use of such terms is described in table \ref{table:priorities} (priorities) and table \ref{table:complexity} (complexity).

\begin{table}[h]
\begin{center}
\begin{tabular}{ | c | p{12.5cm} | }
  \hline
  Priority & Description \\
  \hline\noalign{\smallskip}\noalign{\smallskip}\hline
  High & An essential requirement. The product \textbf{must fulfill} the requirement in order to be satisfactory. \\
  Medium & A useful requirement. The product \textbf{should fulfill} the requirement to maximise effectiveness. \\
  Low & A desiderabe requirement. The product \textbf{could fulfill} the requirement to be more interesting for certain stakeholders. \\
  \hline
\end{tabular}
\end{center}
\caption{Priority descriptions}
\label{table:priorities}
\end{table}

\begin{table}[h]
\begin{center}
\begin{tabular}{ | c | p{12.5cm} | }
  \hline
  Complexity & Description \\
  \hline\noalign{\smallskip}\noalign{\smallskip}\hline
  High & The requirement is difficult to implement. It will take a considerable amount of time. (more than 30 hours?) \\
  Medium & The requirement is moderately difficult. It will take some time. (from 10 to 30 hours?) \\
  Low & The requirement is easy and can be achieved in a short amount of time. (10h or less?) \\
  \hline
\end{tabular}
\end{center}
\caption{Complexity descriptions}
\label{table:complexity}
\end{table}

%-----

\section{Stakeholders}
\label{section:stakeholders}

This section contains the stakeholders of our system.
A short description of the role of each stakeholder is given, and what concerns they might have.

\subsection{Customer}
The customer is the one that is going to guide us to the solution he wants.
His concern is that we should maintain an effective and good communication with him, so he understands our progress.
We should also document the advancement we are making and have a clear system architecture for him.
His main concern is to get a working prototype of our system according to his requirements.

\subsection{Course evaluators}
The course evaluators are the one that are going to evaluate our project. 
Their concern is that the project should be well documented and understandable, and that all the deliveries should be on schedule.
Good communication is also of importance.

\subsection{Developers}
The developers include the creators of the system, and third party developers.
The third party developers are going to integrate our application to their system.
Their concern is that our integration platform should easily be connected with their applications.

\subsection{Users}
The users are the one that are going to use our application.
Their concern is that the system should be easy to use.
It should be easy to add third party applications to the integration platform, and easy to push data to our system.

%-----
\newpage
\section{Funcional requirements}
\label{section:functionalreq}

This section describes the functional requirements for the product.
Each requirements has a priority which helps us identify the focus of our work.
Priorities can be reviewed based on customer's feedback. We tried to prioritize those requirements corresponding to the functionality of the system that the customer expressed most interest about. In order to better organize our workload we also assigned a \'difficulty\' to each requirement.

\textbf{Functional requirements for Integration platform}

\begin{table}[h]
\begin{center}
\begin{tabular}{ | c | p{9cm} | c | c | }
  \hline
  ID & Description & Difficulty & Priority\\
  \hline\noalign{\smallskip}\noalign{\smallskip}\hline
  FIP1	& The IP shall support REST endpoints for receiving heart rate data models expressed as JSON strings	& Med	& High \\
  FIP2	& The IP shall support REST endpoints for receiving weight data models expressed as JSON strings 		& High	& High \\
  FIP3	& The IP shall support REST endpoints for forwarding heart rate models (using JSON) to other systems.	& Med	& High \\
  FPI4	& The IP shall support REST endpoints for forwarding weight models (using JSON).						& Med	& High \\
  \hline
\end{tabular}
\end{center}
\caption{Functional requirements for Integration platform}
\label{table:reqip}
\end{table}

\textbf{Functional requirements for the web frontend}

\begin{table}[h]
\begin{center}
\begin{tabular}{ | c | p{9cm} | c | c |}
  \hline
  ID & Description & Difficulty & Priority\\
  \hline\noalign{\smallskip}\noalign{\smallskip}\hline
  FW1	& The web frontend shall display the data stored by the Integration Platform using charts.	& Med	& High \\
  FW2	& The web frontend shall use Helsenorge color palette										& Low	& Low \\
  \hline
\end{tabular}
\end{center}
\caption{Functional requirements for the web frontend}
\label{table:reqfrontend}
\end{table}

\textbf{Functional requirements for Heart rate application}

\begin{table}[h]
\begin{center}
\begin{tabular}{ | c | p{9cm} | c | c |}
  \hline
  ID & Description & Difficulty & Priority\\
  \hline\noalign{\smallskip}\noalign{\smallskip}\hline
  FHR1	& The application shall measure user’s heart rate using the device camera	& Med	& High \\
  FHR2	& The application shall display acquired measurements on the screen.		& Low	& Low \\
  FHR3	& The application shall forward the data to the IP using its REST endpoint. & Med	& High \\
  \hline
\end{tabular}
\end{center}
\caption{Functional requirements for Heart rate application}
\label{table:reqfrontend}
\end{table}

\textbf{Functional requirements for Weight application}

\begin{table}[h]
\begin{center}
\begin{tabular}{ | c | p{9cm} | c | c |}
  \hline
  ID & Description & Difficulty & Priority\\
  \hline\noalign{\smallskip}\noalign{\smallskip}\hline
  FHV1	& The application shall fetch data from HealthVault.						& Med	& High \\
  FHR2	& The application shall show the user the data it has fetched.				& Low	& Med \\
  FHR3	& The application shall forward the data to the IP using its REST endpoint. & Med	& High \\
  \hline
\end{tabular}
\end{center}
\caption{Functional requirements for Weight application}
\label{table:reqfrontend}
\end{table}

%-----

\section{Non-funcional requirements}
\label{section:nonfunctionalreq}

This section outlines non-functional requirements (quality attributes) for the product.
In order to provide a better overview, we have organized them in categories.

\begin{table}[h]
\begin{center}
\begin{tabular}{ | c | c |p{6.5cm} | c | c |}
  \hline
  ID & Category & Description & Difficulty & Priority\\
  \hline\noalign{\smallskip}\noalign{\smallskip}\hline
  NF1 & Documentation & The system shall be thoroughly documented, both at the code level and by the document ‘project report’.
  & High & High \\
  NF2 & Documentation & Although security and privacy are not requirements for the product they are important topics to be discussed in the documentation.
  & Med & High \\
  NF3 & Open-source	& The product shall be released under a permissive license approved by the product owner.
  & Low & High \\
  NF4 & Interoperability & The system shall provide a good degree of interoperability. Third party application developers should be put in the condition to develop third party (interoperable) solutions rapidly.
  & Med & High \\
  NF5 & Interoperability & A number of two-three prototype applications shall be developed in order to showcase the functionality of the system.
  & Med & High \\
  NF6 & Accessibility & The web-frontend should have a good degree of accessibility. It should have a rather simple design and use a user-provided palette.
  & Low & Low \\
  \hline
\end{tabular}
\end{center}
\caption{Non-functional requirements}
\label{table:reqfrontend}
\end{table}

% -----

\section{Requirements Validation}
(Sommerville 7.3)
How to ensure that a Requirements Document does define 
the customers’ requirements (check against desirable 
features)
• Comprehensibility - do customers understand the 
requirements?
• Verifiability (testability) - can requirements be 
realistically tested? If you cannot test it, it is not a 
requirement!
• Traceability - source of the requirement? Important 
when making changes.
• Adaptability (changeability) - what is the impact of 
changing a particular requirement?

\section{Use cases}



