\chapter{Introduction} 
\label{Introduction} 
\lhead{Chapter 1. \emph{Introduction}}

Write a short text about what this chapter is about.
%---------------------------------------
\section{Project description}

The purpose of the project is to design, develop and document an integration platform for citizen health care.
The intention of such platform is to allow citizens to publish information regarding their health produced by devices in their possession like mobile phones and tablets, as well as fetching such data into third party solutions.

The name we chose for the project is \textbf{NIPEN}. It is simply an acronym for \textbf{N}ational \textbf{I}ntegration \textbf{P}latform for \textbf{e}Health in \textbf{N}orway.

%----------------------------------------------------------------------------------------

\section{The client}

The customer of this project was the Department of the Health portal, Norwegian Directorate of Health (Helsedirektoratet).

%Helsedirektoratet er eit fagdirektorat og myndigheitsorgan som ligg under og blir etatsstyrt av Helse- og omsorgsdepartementet. 
%Helsedirektoratet har også oppgåver frå Kommunal- og regionaldepartementet.

The Directorate has, among other, the task of digitalizing Norway's health care system by providing services for both specialists and citizens.

The customer was represented by Mr. Helge T. Blindheim. His contact is shown in table \ref{table:client}


\begin{table}
\begin{center}
\begin{tabular}{ l | l | l }
  \hline
  Name & Phone & E-mail \\
  \hline\noalign{\smallskip}\noalign{\smallskip}\hline
  Helge T. Blindheim	& 46675321 & Helge.T.Blindheim@helsedir.no \\
  \hline
\end{tabular}
\end{center}
\caption{Customer representative}
\label{table:client}
\end{table}


\section{Involved parties}

The people involved in this project were the customer, the team and the supervisor.
The customer, introduced in the previous section, was represented by Mr. Helge T. Blindheim.
The team consisted of three students from the Department of Computer and Information Science (IDI) at the Norwegian University of Science and Technology (NTNU). Their contact information is shown in table \ref{table:team}. The group was supervised by PhD. candidate Zhu Meng. His contact information is shown in table \ref{table:advisor}.


\begin{table}
\begin{center}
\begin{tabular}{ l | l | l }
  \hline
  Name & Phone & E-mail \\
  \hline\noalign{\smallskip}\noalign{\smallskip}\hline
  Anders Olsen Sandvik	& 91824583 & andsan@stud.ntnu.no \\
  Emanuele Di Santo		& ... & lemrey@gmail.com \\
  Sebastian Zalewski	& 95107928 & zalewski@stud.ntnu.no \\
  \hline
\end{tabular}
\end{center}
\caption{Team members}
\label{table:team}
\end{table}

\begin{table}
\begin{center}
\begin{tabular}{ l | l | l }
  \hline
  Name & Phone & E-mail \\
  \hline\noalign{\smallskip}\noalign{\smallskip}\hline
  Zhu Meng	& 73551189 & zhumeng@idi.ntnu.no \\
  \hline
\end{tabular}
\end{center}
\caption{Student advisor}
\label{table:advisor}
\end{table}


%--------------
\newpage
\section{Project background}

This project is part of the Customer Drivent Project (TDT4290) at NTNU.
Digital healthcare is about using information technologies to provide solutions to problems in healthcare. This definition includes a lot of different domains among which is eHealth. eHealth is
\begin{quote}
an emerging field in the intersection of medical informatics, public health and business, referring to health services and information delivered or enhanced through the Internet and related technologies.\citep{ehealth}
\end{quote}
eHealth projects are therefore long, complex and inherently costly. At the same time the progress in information technology has made available powerful and yet cheap devices which can be used to monitor health. These devices are nowdays widespread and used by a large part of the population.


\section{Problem domain}

\section{Project objective}

The goal of the customer is to investigate the possibilities for national eHealth's projects to leverage the dynamics of the market.

The purpose of the course is to let students acquire practical experience in development of a medium-large software project,
including experience in project management, group dynamics and customer relations.

\section{Duration}
The project started on august 21th and the final presentation is on november 21th. That gives us a total of 13 weeks to work on the project. The instructors specify a workload of 24 hours per week according to the course page[*]. That makes a total of 312 hours per student. We are a group of three students which makes the total 936 hours.

Start date: 21.08.2013
End date : 21.11.2013

[ADD TO BIB] Retrived 25.09.2013 http://www.ntnu.edu/studies/courses/TDT4290/2013
