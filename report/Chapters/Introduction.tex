
\chapter{Introduction} 
\label{Introduction} 

\lhead{Chapter 1. \emph{Introduction}}

This chapter provides a description of the project, its goals, duration, background and involved parties.
At the end, we present a brief description of every other chapter in this document.

%-------------------------
\section{Project NIPEN}
\label{section:description}

The purpose of the project was to design, develop and document a prototype of the National Integration Platform
for Citizen Centric eHealth in Norway.
The intention of such a platform is to enable citizens’ the ability to publish and fetch health information they produce
into the government run citizen centric health portal helsenorge.no.
The project description as provided by the customer can be found in Appendix \ref{AppendixA}.

The name we chose for the project is \textbf{NIPEN}.\newline
It is simply an acronym for \textbf{N}ational \textbf{I}ntegration \textbf{P}latform for \textbf{e}Health
in \textbf{N}orway.

%-------------------------
\section{Helsedirektoratet}
\label{section:customer}

The Directorate of Health, Department of the Health Portal (Helsedirektoratet) was the project's customer.
Their offices are located in Oslo, Norway. 
The Department has, among other responsibilities, the task of digitalizing Norway's health care system by providing
services for both specialists and citizens. The customer was represented by Mr. Helge T. Blindheim.
His contact information is shown in table \ref{table:contact}.

\section{Involved parties}
\label{section:parties}

The people involved in this project were the customer, the supervisor and the team.
The customer, introduced in section \ref{section:customer}, was represented by Mr. Helge T. Blindheim.
The team consisted of three students from the Department of Computer and Information Science (IDI) at the Norwegian University of Science and Technology (NTNU). The group was supervised by PhD. candidate Zhu Meng.
See table \ref{table:contact} for contact information.

\begin{table}[h]
\begin{center}
\begin{tabular}{ | l | l | l | p{2.5cm} | }
  \hline
  Name & Phone & E-mail & Party \\
  \hline\noalign{\smallskip}\hline
  Helge T. Blindheim	& 46675321		& Helge.T.Blindheim@helsedir.no		& Customer\\
  Meng Zhu				& 73551189		& zhumeng@idi.ntnu.no				& Advisor\\
  Anders Olsen Sandvik	& 91824583		& andsan@stud.ntnu.no				& Team member \\
  Emanuele Di Santo		& none			& lemrey@gmail.com					& Team member \\
  Sebastian Zalewski	& 95107928		& zalewski@stud.ntnu.no				& Team member \\
  \hline
\end{tabular}
\end{center}
\caption{Contact information}
\label{table:contact}
\end{table}

%--------------
\section{Background}
\label{section:background}

Digital healthcare is about using information technologies to provide solutions to problems in healthcare.
This definition includes a lot of different domains among which is eHealth:
\begin{quote}
an emerging field in the intersection of medical informatics, public health and business, referring to health services
and information delivered or enhanced through the Internet and related technologies.\citep{ehealth}
\end{quote}
eHealth projects, especially on a national scale, are therefore long, complex and inherently costly.
At the same time the progress in information technology has made available powerful and yet cheap devices which
can be used to monitor people's health. These devices are nowdays widespread and used by a large part of the
population. The Department of Health Portal (our customer) is interested in investigating the
possibilities and advantages of using these devices in national eHealth projects.

%This project is part of the Customer Drivent Project (TDT4290 2013) at NTNU.

\section{Objective}
\label{section:objective}

The project had multiple goals.
The purpose of the course which this project is part of is to let students acquire practical experience in development
of a medium-large software project, including experience in project management, group dynamics and customer relations.

The aim of the customer was to investigate the possibilities for national eHealth projects to use widely available
devices such as smartphones and tables. The customer also requested a prototype system that could demonstrate such
possibilities and stimulate further work.

Expected deliverables:
\begin{itemize}
\item an extensive documentation of the process and the product, called 'Project report'
\item an Integration platform for health information
\item a number of prototype applications using the Integration platform
\end{itemize}

\section{Duration}
\label{section:duration}
The project started on August the 21th. The final presentation will be held on November the 21th.
That makes a total of about 13 weeks to work on the project.
We were advised to dedicate at least 24 hours per week to the project by our teachers.
That makes a total of 312 hours per student. Because we are a group of three students we estimate
the project to take at least 936 hours of work to be completed.

\begin{itemize}
\item Start date: 21.08.2013
\item End date: 21.11.2013
\item Total hours: 936
\item Months: 3
\item Days: 93
\end{itemize}

\section{Report structure}

See below for a summary of each chapter in this document.

\begin{itemize}
	\item \textbf{Project management}\newline
		This chapter contains details about the project plan and organization.
	\item \textbf{Preliminary studies}\newline
		This chapter covers our preliminary studies, including technologies
		and similar solutions. At the end of each section we draw a conclusion
		on each topic covered.
	\item \textbf{Requirements and specification}\newline
		The purpose of this chapter is to formally identify the project stakeholders
		and requirements, both functional and non-functional.
	\item \textbf{System architecture}\newline
		Provides an insight into the architecture of the system,
		detailing every part of it and discussing the rationale behind its design.
	\item \textbf{Security}\newline
		The aim of this chapter is to describe some security practices that
		should be put in place for this type of projects.
	\item \textbf{Sprint 0-5}\newline
		Describe in detail the goals, planning and results of each sprint.
		Every chapter is concluded by an internal evaluation of the sprint.
	\item \textbf{Conclusion and further work}\newline
		In this chapter we state our results and draw a conclusion.
		We point out at the limitations of our product and possible further work
		that can be done to improve it.
	\item \textbf{Evaluation}\newline
		This chapter summarizes our evaluation about every aspect of this project:
		the results, the learning outcome, the social dynamics\ldots
\end{itemize}