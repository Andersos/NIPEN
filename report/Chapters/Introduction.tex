
\chapter{Introduction} 
\label{Introduction} 

\lhead{Chapter 1. \emph{Introduction}}

This chapter provides a description of the project including its background, goals and duration as well as the parties involved.

%-------------------------
\section{Project description}
\label{section:description}

The purpose of the project was to design, develop and document an integration platform for citizen eHealth. \cite{ehealth} % in Norway. 
The aim of such platform is to integrate digital information about citizens' health which is collected using popular devices such as smartphones and tablets.
%citizens to publish data related to their health produced by devices or third party solutions. 
%It is very common today for people to log health data from mobile phones and tablets in their possession.
The project description as provided by the customer can be found in Appendix \ref{AppendixA}.

%% very relevant but to be written elsewhere. (project goal)
%For the client it was important that this project produced a prototype that could be demonstrated to:
%1. Educated health professionals
%2. Developers

%% very relevant but to be written elsewhere. (project limitations?)
% Security is of high importance when dealing with citizen health data. This was not made a requirement of the project beacue the assignment had to be scaled down because the group only consisted of three members. We will however discuss how you could add a secure layer to the working solution. 

The name we chose for the project is \textbf{NIPEN}.\newline
It is simply an acronym for \textbf{N}ational \textbf{I}ntegration \textbf{P}latform for \textbf{e}Health in \textbf{N}orway.

%-------------------------
\section{Project sponsor}
\label{section:client}

The project customer was the Department of the Health portal, Norwegian Directorate of Health (Helsedirektoratet) located in Oslo, Norway.
The Directorate has, among other, the task of digitalizing Norway's health care system by providing services for both specialists and citizens. The customer was represented by Mr. Helge T. Blindheim. His contact information is shown in table \ref{table:contact}.  
%Their office is located in the capital of Norway, Oslo.

% --- relevant but to be detailed in another chapter (proj. mgmt) - emanuele
%This infered that our weekly meeting had to be heald over teleconferencing.

%% COMMENTED OUT
\iffalse
\begin{table}[h]
\begin{center}
\begin{tabular}{ | l | l | l |  l | }
  \hline
  Name & Phone & E-mail & \\
  \hline\noalign{\smallskip}\noalign{\smallskip}\hline
  Helge T. Blindheim	& 46675321 & Helge.T.Blindheim@helsedir.no \\
  \hline
\end{tabular}
\end{center}
\caption{Customer representative}
\label{table:client}
\end{table}
\fi


\section{Involved parties}
\label{section:parties}

The people involved in this project were the customer, the team and the supervisor.
The customer, introduced in section \ref{section:client}, was represented by Mr. Helge T. Blindheim.
The team consisted of three students from the Department of Computer and Information Science (IDI) at the Norwegian University of Science and Technology (NTNU). The group was supervised by PhD. candidate Zhu Meng.
See table \ref{table:contact} for contact information.

\begin{table}[h]
\begin{center}
\begin{tabular}{ | l | l | l | l | }
  \hline
  Name & Phone & E-mail & Party \\
  \hline\noalign{\smallskip}\noalign{\smallskip}\hline
  Helge T. Blindheim    & 46675321    & Helge.T.Blindheim@helsedir.no & Product owner\\
  Zhu Meng              & 73551189    & zhumeng@idi.ntnu.no           & Student advisor\\
  Anders Olsen Sandvik	& 91824583   & andsan@stud.ntnu.no            & Team member \\
  Emanuele Di Santo     & none       & lemrey@gmail.com               & Team member \\
  Sebastian Zalewski    & 95107928   & zalewski@stud.ntnu.no          & Team member \\
  \hline
\end{tabular}
\end{center}
\caption{Contact information}
\label{table:contact}
\end{table}

%% COMMENTED OUT
\iffalse
\begin{table}[h]
\begin{center}
\begin{tabular}{ | l | l | l | }
  \hline
  Name & Phone & E-mail \\
  \hline\noalign{\smallskip}\noalign{\smallskip}\hline
  Zhu Meng	& 73551189 & zhumeng@idi.ntnu.no \\
  \hline
\end{tabular}
\end{center}
\caption{Student advisor}
\label{table:advisor}
\end{table}
\fi


%--------------
\newpage
\section{Project background}
\label{section:background}

This project is part of the Customer Drivent Project (TDT4290) at NTNU.
Digital healthcare is about using information technologies to provide solutions to problems in healthcare. This definition includes a lot of different domains among which is eHealth. eHealth is
\begin{quote}
An emerging field in the intersection of medical informatics, public health and business, referring to health services and information delivered or enhanced through the Internet and related technologies.\citep{ehealth}
\end{quote}
eHealth projects of national scale are therefore long, complex and inherently costly. At the same time the progress in information technology has made available powerful and yet cheap devices which can be used to monitor people's health. These devices are nowdays widespread and used by a large part of the population. As such, the Department of Health Portal is interested in investigating the possibilities of using these devices in eHealth projects.

%The objective of this project is to design, document and develop a solution that integrates the data gatherd from these devices and display a complete collection of the users health data. Together with educated health professonals the gatherd data could be used to get a better understanding of the users health.

\section{Project objective}
\label{section:objective}

The project had multiple goals.
The purpose of the course which this project is part of is to let students acquire practical experience in development of a medium-large software project, including experience in project management, group dynamics and customer relations.

The aim of the customer was to investigate the possibilities for national eHealth's projects to use widely available devices such as smartphones and tables. The customer was also requested a prototype system that could demonstrate such possibilities and stimulate further work.

\section{Duration}
\label{section:duration}
The project started on August the 21th. The final presentation will be held on November the 21th. That makes a total of 13 weeks (rounded down) to work on the project. We were advised to dedicate at least 24 hours per week to the project by our teachers. %according to the course page[*].
That makes a total of 312 hours per student. Because we are a group of three students we estimate the project to take at least 936 hours of work to be completed.

\begin{itemize}
\item Start date: 21.08.2013
\item End date: 21.11.2013
\item Total hours: 936
\item Months: 3 (exactly)
\item Days: 93
\end{itemize}



%[ADD TO BIB] Retrived 25.09.2013 http://www.ntnu.edu/studies/courses/TDT4290/2013


\section{Summary}

See below for a summary of each chapter in this document.

\textbf{Project management}\newline
...
\textbf{Preliminary studies}\newline
...
\textbf{Requirements and specification}\newline
...